\documentclass{beamer}


% Add
%  -- generation of authentication key (H)
%  -- field & ring definitions

\usepackage{graphicx}
\usepackage{listings}
\usepackage{color}
\usepackage[utf8]{inputenc}

\definecolor{mygreen}{rgb}{0,0.6,0}
\definecolor{mygray}{rgb}{0.5,0.5,0.5}
\definecolor{mymauve}{rgb}{0.58,0,0.82}

\lstset{
backgroundcolor=\color{white},   % choose the background color
basicstyle=\footnotesize\ttfamily,
  breaklines=true,                 % automatic line breaking only at whitespace
  captionpos=b,                    % sets the caption-position to bottom
  commentstyle=\color{mygreen},    % comment style
  escapeinside={\%*}{*)},          % if you want to add LaTeX within your code
  keywordstyle=\color{blue},       % keyword style
  stringstyle=\color{mymauve},     % string literal style
}

\title{Introduction to radare2 and symbolic execution}
\author{Mathias Hall-Andersen (rot256)}
\institute{Pwnies @ Copenhagen University}
\date{2018}

\begin{document}

\frame{\titlepage}

\begin{frame}
\frametitle{Before we start : Get the software}
\textbf{Instructions at} \\
{\small \url{github.com/Pwnies/workshops/tree/master/workshops/r4ge}}
\end{frame}

\begin{frame}
\frametitle{Plan}
\begin{enumerate}
    \item Static analysis using r2 and introduction to r2pipe
    \item Handout crackme
    \item Work session (static analysis of the first stage of the crackme)
    \item Dynamic analysis using r2
    \item Work session (extracting the second stage of the crackme)
    \item Symbolic execution using angr and r4ge
    \item Work session (solving the final stage of the crackme)
\end{enumerate}
\end{frame}

\begin{frame}
\frametitle{Radare2 : Static analysis}




\end{frame}

\begin{frame}
\frametitle{Work session 1 : Static analysis}
Obtain the crackme: \\

\vspace{3mm}

\texttt{curl --output ./obf https://rot256.io/obf} \\
\texttt{curl --output ./obf https://rot256.io/obf-hard}

\vspace{3mm}

Then \texttt{r2 -d ./obf}

\vspace{3mm}

Cheatsheet (all commands you should need):

\vspace{3mm}

\begin{tabular}{l | l}
    Run analysis         & aaa \\
    Help page            & af? (and a?) \\
    Disassemble function & pdf sym.main \\
    Visual mode          & VV, t / f / TAB / hjkl to navigate graph \\
    Rename variable      & afvn [old name] [new name] \\
    Command in visual mode & Shift + :
\end{tabular}

\vspace{3mm}

\textbf{Goal:} Uncover the functionality in "main".

\end{frame}

\begin{frame}
\frametitle{Radare2 : Dynamic analysis}
\end{frame}

\begin{frame}
\frametitle{Work session 2 : Dynamic analysis}

\textbf{Catch up:} \url{https://rot256.io/outer.c}

\vspace{3mm}

\begin{tabular}{l | l}
    Help page        & d? \\
    Restart program  & doo \\
    Set breakpoint   & db [addr] \\
    Continue         & dc \\
    Step             & s / S \\
\end{tabular}

\vspace{3mm}

You might have to: "\texttt{e asm.bits = 64}". \\

\vspace{3mm}

\textbf{Goal:} Extract / inspect inner code (encrypted function):
\end{frame}

\begin{frame}[fragile]
\frametitle{Symbolic execution; a simple example}
\begin{lstlisting}[language=C]
int main(int argc, char* argv[]) {
    if (argc != 2) return -1;

    char *a = argv[1];

    int ok = 1;
    ok &= (a[0] == 1);
    ok &= (a[1] == 1);
    for (int n = 2; n < 10; n++)
        ok &= (a[n] == a[n - 1] + a[n - 2]);

    if (ok) printf("fibulous\n"); // <- how?
}
\end{lstlisting}
\end{frame}

\begin{frame}[fragile]
\frametitle{Symbolic execution; angr, a simple example}
\begin{lstlisting}[language=python]
import angr
import claripy

arg1 = claripy.BVS('arg1', 8 * 10)
proj = angr.Project('./fib')

st = proj.factory.entry_state(args=['./fib', arg1])
sm = proj.factory.simulation_manager(st)
sm.explore()

print sm.deadended

for state in sm.deadended:
    a = state.solver.eval(arg1, cast_to=str)
    print a.encode('hex'), map(ord, a)
\end{lstlisting}
\end{frame}


\begin{frame}[fragile]
\frametitle{Symbolic execution; a more complicated example}
\begin{lstlisting}[language=C]
struct node_t {
    int v; struct node_t* l; struct node_t* r;
} node_t;

int walk(struct node_t* n, int k) {
    if (n->v > k && n->l != 0) return walk(n->l, k);
    if (n->r != 0) return walk(n->r, k);
    return n->v;
}

int check(int value) {
    struct node_t d = { .v =  42, .l = 0, .r = 0};
    struct node_t c = { .v =   1, .l = 0, .r = 0};
    struct node_t b = { .v = -10, .l = 0, .r = &d};
    struct node_t a = { .v =   0, .l = &b, .r = &c};
    return walk(&a, value) == 42; // <- how?
}

int main(int argc, char* argv[]) {
    return check(*((int*)argv[1]));
}
\end{lstlisting}
\end{frame}

\begin{frame}[fragile]
\frametitle{Symbolic execution; angr, a more complicated example}
\begin{lstlisting}[language=python]
import angr
import struct
import claripy

value = claripy.BVS('value', 32)
proj  = angr.Project('./tree')
check = proj.loader.find_symbol('check')
st    = proj.factory.call_state(
    check.rebased_addr, value, ret_addr=0xdeadbeef)
simgr = proj.factory.simgr(st)

while simgr.active:
    for s in simgr.active:
        if s.addr == 0xdeadbeef:
            s.solver.add(s.regs.eax == 1)
            try:
                v = s.solver.eval(value, cast_to=str)
                print struct.unpack('i', v)[0]
            except:
                pass
    simgr.step()
\end{lstlisting}
\end{frame}

\begin{frame}
\frametitle{Symbolic execution; r4ge}
\end{frame}

\begin{frame}
\frametitle{Work session 3 : Symbolic execution}

\textbf{Catch up (source of inner):} \url{https://rot256.io/inner.c}

\vspace{3mm}

Two options:

\begin{enumerate}
    \item Use \texttt{r2pipe} to extract the code and use angr from python.
    \item Use the \texttt{r4ge} plug-in directly from r2.
\end{enumerate}

\end{frame}

\end{document}
