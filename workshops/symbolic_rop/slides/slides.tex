\documentclass{beamer}

\usepackage{graphicx}
\usepackage{listings}
\usepackage{color}
\usepackage[utf8]{inputenc}

\definecolor{mygreen}{rgb}{0,0.6,0}
\definecolor{mygray}{rgb}{0.5,0.5,0.5}
\definecolor{mymauve}{rgb}{0.58,0,0.82}

\lstset{
backgroundcolor=\color{white},   % choose the background color
basicstyle=\footnotesize\ttfamily,
  breaklines=true,                 % automatic line breaking only at whitespace
  captionpos=b,                    % sets the caption-position to bottom
  commentstyle=\color{mygreen},    % comment style
  escapeinside={\%*}{*)},          % if you want to add LaTeX within your code
  keywordstyle=\color{blue},       % keyword style
  stringstyle=\color{mymauve},     % string literal style
}

\title{Automatic ROP chain generation}
\author{Jonas 'Kokjo' Rudloff }
\institute{Pwnies @ Copenhagen University}
\date{2018}

\begin{document}


\frame{\titlepage}


\begin{frame}
    \frametitle{Before we start : Get the software}
    \textbf{Instructions at} \\
    {\small \url{github.com/Pwnies/workshops/tree/master/workshops/symbolic\_rop}}
\end{frame}


\begin{frame}
    \frametitle{Plan}
    \begin{itemize}
        \item Crash course in ROP for x86, amd64, arm, and mips.
        \item Exercise 1: Manually write some rop chains.
%        \item Introduction to Unicorn.
%        \item Exercise 2: Execute some code using Unicorn.
        \item Introduction to z3 and symbolic execution.
        \item Exercise 3: Solve simple problems using z3.
        \item Using z3 and Unicorn to automagically generating ROP chains.
        \item Exercise 4: Repeat exercise 1 using high powered machinery.
    \end{itemize}
\end{frame}


\begin{frame}
    \frametitle{ROP: Attack scenario}
    \begin{itemize}
        \pause \item Have: A buffer overflow on the stack
        \pause \item Want: The ability to call arbitary functions with arbitary arguments.
    \end{itemize}
\end{frame}


\begin{frame}
    \frametitle{ROP: General strategy}
    \begin{enumerate}
        \pause \item Chain small pieces of code together to manipulate cpu state.
        \pause \item Make state look like a function call, jump to disired function.
        \pause \item Repeat...
    \end{enumerate}
\end{frame}


\begin{frame}
    \frametitle{ROP: x86 example}

    \pause Calling convention:
    \begin{enumerate}
        \item Push arguments to stack(last argument first).
        \item Use \texttt{Call} instruction to push return address and jump to function.
    \end{enumerate}
    
    \pause Stack layout on function entry
    \begin{center}
    \begin{tabular}{l|c}
        ...     & More arguments \\
        ESP + 8 & Argument 2 \\
        ESP + 4 & Argument 1 \\
        ESP + 0 & Return address \\
    \end{tabular}
    \end{center}
\end{frame}


\begin{frame}
    \frametitle{ROP: x86 example (cont.)}
    Call \texttt{foo(1337, 42)} then \texttt{bar(31337)}.
    \pause
    \begin{center}
    \begin{tabular}{l|c}
        ESP + 24 & 31337 \\
        ESP + 20 & return address of bar \\
        ESP + 16 & \texttt{bar} \\
        ESP + 12 & 42 \\
        ESP + 8  & 1337 \\
        ESP + 4  & cleanup gadget: \texttt{pop eax; pop ebx; ret} \\
        ESP + 0  & \texttt{foo}
    \end{tabular}
    \end{center}
\end{frame}


\begin{frame}
    \frametitle{ROP: another x86 example}
    Call \texttt{gets(buffer)} then \texttt{system(buffer)}
    \pause
    \begin{center}
    \begin{tabular}{l|c}
        ESP + 12 & \texttt{buffer} \\
        ESP + 8  & \texttt{buffer} \\
        ESP + 4  & \texttt{system} \\
        ESP + 0  & \texttt{gets}
    \end{tabular}
    \end{center}
    \pause We can 'compress' certain call sequences...
\end{frame}


\begin{frame}
    \frametitle{ROP: amd64 example}
    Calling convention:
    \begin{enumerate}
        \item Argument 1 in \texttt{RDI}
        \item Arugment 2 in \texttt{RSI}
        \item Argument 3 in \texttt{RDX}
        \item Other arguments: some in registres, rest on the stack...\footnote{Not really relevant... Who calls functions with more than 3 arguments anyway?}
        \item Return address on stack.
    \end{enumerate}
    \pause For example to call \text{foo(42, 1337)}
    \pause we must have \texttt{RDI = 42}
    \pause and \texttt{RSI = 1337}
    \pause and the stack looking like:
    \begin{center}
    \begin{tabular}{l|c}
        RSP + 0  & \texttt{Return address of foo}
    \end{tabular}
    \end{center}
\end{frame}


\begin{frame}
    \frametitle{ROP amd64 example (cont.)}
    Call \texttt{foo(1337, 42)} then \texttt{bar(31337)}.
    \begin{center}
    \begin{tabular}{l|c}
        RSP + 56 & \texttt{bar} \\
        RSP + 48 & 31137 \\
        RSP + 40 & Gadget: \texttt{pop rdi; ret} \\
        RSP + 32 & \texttt{foo} \\
        RSP + 24 & 42 \\
        RSP + 16 & Gadget: \texttt{pop rsi; ret} \\
        RSP + 8  & 1337 \\
        RSP + 0  & Gadget: \texttt{pop rdi; ret} \\
    \end{tabular}
    \end{center}
\end{frame}


\begin{frame}
    \frametitle{ROP - exercises}
    \begin{center}
        {\Huge Exercises!}
    \end{center}
    Have a look at \texttt{rop\_playground.*} \\
    Hint: Use \texttt{ROPgadget} to find gadgets.
\end{frame}


\begin{frame}[fragile]
    \frametitle{SAT sovler}

    \begin{enumerate}
        \pause \item We have some varibles (eg. $x$, $y$ and $z$)
        \pause \item ... and some constraints (eg. $x > y z$ and $x < -10$)
        \pause \item and we can concrete values for $x$ and $y$.
    \end{enumerate}

    \pause
    \begin{lstlisting}[language=python]
from z3 import *
x = Int("x")
y = Int("y")
z = Int("z")
solver = Solver()
solver.add(x > y*z)
solver.add(x < -10)
assert solver.check() == sat
print solver.model()
    \end{lstlisting}

    \pause
    \begin{lstlisting}
$ python z3_example.py 
[x = -11, z = 1, y = -12]
    \end{lstlisting}
\end{frame}


\begin{frame}
    \frametitle{How z3 works internally}
    \pause 
    \begin{figure}[h]
        \includegraphics[width=0.75\textwidth]{z3doge}
    \end{figure}
\end{frame}


\begin{frame}
    \frametitle{z3 - exercises}
    \begin{center}
        {\Huge Exercises!}
    \end{center}
    Have a look at \texttt{solve\_sudoku.py}
\end{frame}


\begin{frame}
    \frametitle{Automatic ROP}
    Rough sketch
    \begin{itemize}
        \item Find gadgets using the \texttt{ROPgadget} tool.
        \item Analyse gadgets with Unicorn engine
        \item Use z3 to 'chain' gadgets together
        \item Use z3 to choose which gadgets to chain together.
    \end{itemize}
\end{frame}

\end{document}
